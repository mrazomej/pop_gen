\subsubsection{Mutation}

One of the ingredients for evolution to take place is the constant appearance
of genetic variability. After all, the raw material for evolution to act on is
the appearance of new mutations in the population. This is what we will model
now for our one-locus two-allele case. It might be a little counterintuitive at
first that the mutation term must be include as part of our directional term in
the diffusion equation $M(f)$ since it is common to think of mutations at
random events with no particular direction. But if we think of mutations
between both alleles $A$ and $a$ as a simple first-order chemical reaction of
the form
\begin{equation}
  A \xrightleftharpoons[\mu_{A\rightarrow a}]
  {\,\mu_{a\rightarrow A}\,} a,
\end{equation}
where $\mu_{A\rightarrow a}$ is the mutation rate from $A$ to $a$ and 
$\mu_{a\rightarrow A}$ is the mutation rate from $a$ to $A$, we can see that 
mutation must be a directional term. For example if the forward rate 
$\mu_{A\rightarrow a}$ was much larger than the reverse rate 
$\mu_{a\rightarrow A}$, then this would push the population towards high $A$
frequency. Even if the mutation rates were equal, this would still direct the
population towards an interemediate allele frequency.