\section{Continuous-time continuous-state Markov process}

The mathematical language to study diffusion theory, as with many other field of
science is the language of probability theory. As Jaynes' famous book title
suggested, \textit{probability theory is the logic of science}. It is through
this formal language that we can assess how likely are specific events to
happen. For example, later on we'll see that within the framework of diffusion
theory is possible to ask what is the probability of an allele being fixed in
the population given the evolutionary forces acting on it.

Of particular interested for our endeavor is the specific mathematical objects
known as Markov processes. The definition of these processes is very simple.
First of all by a ``process'' what we mean is a series of events that evolve
over time. For example we will be studying how the frequency of a particular
allele $f$ changes over time as the population evolves. The Markovian property -
named after Andrey Markov, a famous Russian mathematician from the XIX century -
can be stated as follows:
\begin{tcolorbox}
  A Marvov process is a type of stochastic process for which the transition
  states only depends on the current state.
\end{tcolorbox}
To gain intuition for what this means imagine we are measuring a variable $x$
over time such that we have a bunch of pairs of the form $(x_1, t_1), (x_2,
t_2), (x_3, t_3), \ldots (x_{n+1}, t_{n+1})$. For a Markov process, the
probability of ending at state $x_{n+1}$ at time $t_{n+1}$ given all the
previous visited states is of the form
\begin{equation}
  P(x_{n+1}, t_{n+1} \mid x_1, t_1; x_2, t_2; \ldots; x_n, t_n) =
  P(x_{n+1}, t_{n+1} \mid x_n, t_n).
  \label{eq_chapman_kolmogorov}
\end{equation}
In other words, knowing the entire history of variable $x$ as it evolves over
time does not help us predict the next time step. All we need to know is the
last position. Markov processes are usually called memoryless stochastic process
because of this property that the process doesn't remember the full trajectory
when it ``decides'' where to move in the next step. Assuming a Markovian process
is arguably one of the most widely used assumptions in all of science. From
chemical reactions, to brownian motion of a particle, to our particular case of
interest of allele frequencies changing over time.

In the classic formulation of diffusion theory we are concerned with a
particular type of Markov processes. Diffusion theory works in the limit of
large populations where we can assume that the frequency of a particular allele
rather than being a discrete entity that increases in factors of $1 / N$ where
$N$ is the number of organisms, is a continuous value in the range between 0
and 1. We also assume that we can compute changes in this frequency in
continuous time. That means that both $f$ the frequency of an allele, and $t$
the time are continuous variables. Therefore our stochastic process are
described with the creative name of continuous-time continuous-state Markov
processes. We will now state a very important equation for this type of
Markovian processes known as the Chapman-Kolmogorov equation.

\subsection{Chapman-Kolmogorov equation}

The Chapman-Kolmogorov equation is an important property of continuous-time
continuous-state Markov processes that we will use for many of our derivations
in the coming sections. The equation is stated as follows: For three time points
$t_1 < t_2 < t_3$ for which we measured a stochastic variable $x$ we have that
\begin{equation}
  P(x_3, t_3 \mid x_1, t_1) = \int dx_2\; P(x_3, t_3 \mid x_2, t_2)
                                          P(x_2, t_2 \mid x_1, t_1),
\end{equation}
where the integral is taken over the domain of values that the random variable
$x$ can take. In other words, to calculate the transition probability between
$x_1$ and $x_3$ with an intermediary step $x_2$, we must add (integrate for
continuous variables) all possible values that $x_2$ can take. This concept is
schematically represented in \fref{fig01_01}

\begin{figure}[h!]
	\centering \includegraphics[width=0.5\textwidth]
  {./fig/chapter_prob/01_01.png}
	\caption{\textbf{Schematic of the Chapman-Kolmogorov equation}. The
  Chapman-Kolmogorov equation is a statement about the transition between two
  points $x_1$ and $x_3$, adding all possible intermediary steps $x_2$.}
  \label{fig01_01}
\end{figure}

In the coming section we will use the Chapman-Kolmogorov equation to derive the
so-called continuous master equation. This will be the foundation from which we
will get to the main results of diffusion theory.
