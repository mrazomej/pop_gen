\section{Fokker-Planck equation in population genetics}

In the last section after we performed a Taylor expansion of the master equation
we ended up with an equally complicated partial differential equation of
infinite order. So in order to make any progress towards simplifying the
treatment of the equation what we will do is trade accuracy for simplicity.
More concretely if our assumptions on the shape of the transition probability
$\phi_t(f; r)$ being sharply peaked are to be taken seriously, we can then
truncate the Kramers-Moyal expansion to include only up to second order
derivatives. This truncated equation would locally approximate the time
evolution of the probability distribution $P(f, t)$. \eref{eq_km_expansion} then
is of the form
\begin{equation}
  \ddt{P(f, t)} = - {\partial \over \partial f}
  \left[
  a^{(1)}(f, t) P(f, t)
  \right] +
  {1 \over 2}{\partial^2 \over \partial f^2}
  \left[
  a^{(2)}(f, t) P(f, t)
  \right].
  \label{eq_fokker_planck}
\end{equation}
In the physics literature this is known as the Fokker-Planck equation, while in
the mathematics literature is known as the Kolmogorov forward equation. One
intriguing aspect of how to arrive to this equation is the seemingly arbitrary
choice of truncating up to the second moment. The argument that is often thrown
around is that the ``art'' of these truncations is to stop at the first
non-vanishing moment. But for the specific case of the Kramers-Moyal expansion
of the master equation there is a theorem - Pawula theorem - that shows that for
the solutions of the Kramers-Moyal expansion to be interpreted as probability
densities the expansion must either contain one, two or an infinite number of
moments. So two sounds much better than infinite, doesn't it?
\mrm{Need to include appendix with the proof of Pawula theorem.}

\subsection{Determination of the Fokker-Planck coefficients}

The two jump moments in \eref{eq_fokker_planck}, $a^{(1)}$ and $a^{(2)}$ defined
by \eref{eq_jump_mom} have a specific interpretation in population genetics.
Here is where a little bit of a terminology conflict between physics and
evolutionary biology comes into play. The directional term, i.e. the one with
the first order derivative in \eref{eq_fokker_planck} is known in the physics
literature as the drift term on a diffusion-like equation. This is confusing
because in evolutionary theory the diffusive term, i.e. the one with the second
derivative in \eref{eq_fokker_planck} is known as the genetic drift term. I will
try to be as consistent as possible on these notes using the terms directional
and diffusive to avoid confusion.

Coming back to the directional term, we define $M(f, t)$ to be
\begin{equation}
  M(f, t) \equiv \ee{r(t)} = \int_{-\infty}^{\infty} dr \; \phi_t(f; r) r,
\end{equation}
i.e. the mean of the jump distribution. As we will see in coming sections this
term captures the effect of directional evolutionary forces such as selection,
mutation and migration. For the diffusive term we define $V(f, t)$ as the second
moment of the jump distribution
\begin{equation}
  V(f, t) \equiv \ee{r^2(t)} = \int_{-\infty}^{\infty} dr \; \phi_t(f; r) r^2.
\end{equation}
This term captures the random sampling of alleles, also known as genetic drift.
In all of the population genetics literature I have encounter so far, this term
$V(f, t)$ is treated as the \textbf{variance} rather than the second moment.
This is partly because computing the specific functional form of the variance
for different models of reproduction is much simpler. The variance
$\sigma_r^2(t)$ is defined as
\begin{equation}
  \sigma_r^2(t) = \ee{\left( r - \ee{r} \right)^2} = \ee{r^2} - \ee{r}^2.
\end{equation}
We can therefore work with this more convenient quantity if we assume that
$\ee{r}^2 \approx 0$. This is a reasonable assumption given that for the
Fokker-Planck equation to be accurate we assumed a tight distribution for the
jumpt size $\phi_t(f; r)$. The peaked nature of this distribution must imply
that $\ee{r} \ll 1$, but more importantly, we will assume that
$\ee{r}^2 \ll \ee{r^2}$. So upon using these two definitions we arrive to one
of the main results in population genetics, the Kimura diffusion equation
\begin{equation}
  \ddt{P(f, t)} = - {\partial \over \partial f}
  \left[
  M(f, t) P(f, t)
  \right] +
  {1 \over 2}{\partial^2 \over \partial f^2}
  \left[
  V(f, t) P(f, t)
  \right].
  \label{eq_kimura_diffusion}
\end{equation}
