% !TEX root = ../../main.tex
\subsection{Modeling the forces of evolution}

The strength of diffusion theory mainly comes from the ability of being able to
combine all forces of evolution - mutation, migration, selection and genetic
drift - on a single theoretical description. Now that we have derived the
equilibrium distribution of allele frequency for our one-locus two-allele system
let's propose models for the functional form of the directional ($M(f)$) and the
diffusive term ($V(f)$). We will focus on three forces, leaving migration out of
the picture for now.

\subsubsection{Frequency independent selection (haploids)}

Because of our personal love for bacteria and the fact that many of the most
interesting experiments in evolution have been and are being done in haploid
organisms, we will first focus our efforts on these single-cell organisms.
Haploidy refers to the biological feature of having a single copy of a
chromosome. As bacteria divide into two cells they segregate (split) one copy of
the genome to each of the daughter cells (extrachromosomal elements such as
plasmids behave differently). This in contrast with diploidy where organisms
such as humans that receive one copy of each chromosome form each parent.

What haploidy means for our task of modeling selection on a one-locus two-allele
system is that we only need to focus on the fitness value of each of the alleles
without having to worry about combination of alleles. The concept of fitness,
despite of being in the daily jargon of many biologists, is an elusive term not
that easy to define. Roughly speaking we can think of fitness as the ability of
an organism, or a population of organisms to survive and reproduce in the given
ecological niche they occupy. The term ecology has to be included because
fitness is a result of the interplay between organisms with their environment
including all biotic and abiotic interactions. Having said that we will first
begin with the simplest form of fitness, i.e. frequency independent selection.
The term frequency independence simply refers to the assumption that the fitness
of a particular allele does not depend on the relative abundance of such allele.
This assumption could break down for cases such as some pathogenic bacteria that
coordinate their attack via cell-to-cell communication known as quorum sensing.
But we need to learn how to crawl before we decide to run.

Let's assume that we have two versions of an allele $A$ and $a$. We define $f$
to be the frequency of allele $A$ in the population, leaving $1 - f$ to be the
frequency of allele $a$. Let us now assign fitness values $\omega_A$ and
$\omega_a$ for each of the alleles. The way to interpret this parametrization of
fitness is as the expected number of individuals carrying an allele in the next
time point. So if at time $t$ we have $N_A(t)$ organisms carrying allele $A$,
for a time $t + \Dt$ we expect to have $N_A(t + \Dt) = N_A(t) \omega_A$
organisms. In other words we assume exponential growth of the organisms of the
form
\begin{equation}
  \dt{N_A} = N_A(t) \omega_A,
\end{equation}
with an equivalent equation for $N_a(t)$. At any point in time the average
fitness $\bar{\omega}$ is then given by
\begin{equation}
  \bar{\omega}(t) = f(t) \omega_A + (1 - f(t)) \omega_a.
\end{equation}

We can compute the allele frequency $f$ at time $t + \Dt$ as
\begin{equation}
  f(t + \Dt) = {N_A(t) \omega_A \over N_A(t) \omega_A + N_a(t) \omega_a}.
\end{equation}
If we multiply and divide this by $N\tot \equiv N_A + N_a$, the total number of
organisms we then have
\begin{equation}
  f(t + \Dt) = {f(t) \omega_A \over f(t) \omega_A + (1 - f(t)) \omega_a},
  \label{eq_f_tdt}
\end{equation}
since $f = N_A/N\tot$ and $(1 - f) = N_a/N\tot$. Notice that the denominator in
\eref{eq_f_tdt} is no other than the average fitness $\bar{\omega}$, so we can
then write
\begin{equation}
  f(t + \Dt) = {f(t) \omega_A \over \bar{\omega}(t)}.
\end{equation}
The change in allele frequency $\Delta f(t, \Dt)$ is then given by
\begin{equation}
  \Delta f(t, \Dt) = f(t + \Dt) - f(t) = {f(t) \omega_A \over \bar{\omega}(t)}
  - f(t) = {f(t) (\omega_A - \bar{\omega}) \over \bar{\omega}}.
\end{equation}
If we substitute the definition of $\bar{\omega}$ for the numerator we then have
\begin{equation}
  \Delta f(t, \Dt) = {f(t) (\omega_A - f(t) \omega_A -
  (1 - f(t)) \omega_a) \over
  \bar{\omega}}.
\end{equation}
This can be simplified as
\begin{align}
  \Delta f(t, \Dt) &= {f(t) \omega_A - f^2(t)\omega_A
  - f(t) \omega_a + f^2(t) \omega_a \over \bar{\omega}(t)}, \\
  &= {\omega_A f(t)(1 - f(t)) - \omega_a f(t) (1 - f(t)) \over
  \bar{\omega}(t)}, \\
  &= {f(t)(1 - f(t))(\omega_A - \omega_a) \over \bar{\omega}(t)}.
\end{align}
Let's take a look at how the mean fitness $\bar{\omega}$ changes as the
frequency $f$ changes, i.e.
\begin{equation}
  {d \bar{\omega} \over df} = {d \over df }
  \left(f \omega_A(t) + (1 - f) \omega_a(t)  \right) =
  \omega_A - \omega_a.
\end{equation}
With this result we can rewrite the change in allele frequency $\Delta f$ as
\begin{equation}
  \Delta f = {f (1 - f) \over \bar{\omega}} {d \bar{\omega} \over df} =
  f (1 - f) {d \ln \bar{\omega} \over df}.
\end{equation}
This is the functional form we will plug into our directional term $M(f)$ to
capture the effects of selection on the allele distribution.

\subsubsection{Mutation}

One of the ingredients for evolution to take place is the constant appearance
of genetic variability. The raw material for evolution to act on is the
appearance of new mutations in the population. This is what we will model now
for our one-locus two-allele case. Might be a little counterintuitive at first
that the mutation term will be include as part of our directional term in the
diffusion equation $M(f)$ since it is common to think of mutations at random
events with no particular direction. But if we think of mutations between both
alleles $A$ and $a$ as a simple first-order chemical reaction of the form
