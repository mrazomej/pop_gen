% !TEX root = ../../main.tex
\section{Kramers-Moyal expansion of the Master equation}

For the most part master equations are hard to deal with beacause of their
integro-differential nature. A way around this issue is to simplify the master
equation using Taylor expansions. This expansion goes by name of Kramers-Moyal
expansion, and the truncated version of the expansion is known in physics
literature as the Fokker-Planck equation, and in the probability theory
literature as the Kolmogorov forward equation. Regardless of the fancy names
behind these manipulations the idea is very straight-forward and very common in
physics in general: we take a complicated function and use its derivatives to
come up with an approximation that is good enough at least locally.

To perform the expansion we need to express the transition probabilities
$\phi_t(f \mid f')$ in \eref{eq_master_eq_full} as a function of the jump size
$r \equiv f - f'$ rather than as conditional probabilities. We therefore express
the transition probabilities as
\begin{equation}
  \phi_t(f \mid f') = \phi_t(f'; r).
  \label{eq_trans_jump_size}
\end{equation}
By performing this change of variables for the integral in
\eref{eq_master_eq_full} we need to compute the Jacobian which is of the form
\begin{equation}
  dr = df' {dr \over df'} = -df'.
\end{equation}

Furthermore since the integration now is done over values of $r$ rather than
$f'$ we must update the integration limits as well. We then have for the lower
limit $f' = 0$ that upon substitution on the definition of $r$ this gives
\begin{equation}
  r(f' = 0) = f - 0 = f.
\end{equation}
Doing the same for the upper limit of $f' = 1$ results in
\begin{equation}
  r(f' = 1) = f - 1.
\end{equation}
Using these changes results in a master equation of the form
\begin{equation}
  \dt{P(f, t)} = \int_f^{f - 1} \underbrace{(-1)}_{\text{Jacobian}} dr \;
  \left[
  \phi_t(f - r; r) P(f - r, t) -
  \phi_t(f; -r) P(f, t)
  \right].
\end{equation}
For the second term inside the square brackets we wrote $\phi_t(f; -r)$ due to
the change of reference point. Since the original term was $\phi_t(f' \mid f)$
that means that the jump took place from $f$ to $f'$ instead of the other way,
therefore the jump rather than being of size $r$ it must be of size $-r$ to
compensate for this change. Flipping the limits of integration because of the
negative sign that the Jacobian gave results in
\begin{equation}
  \dt{P(f, t)} = \int^f_{f - 1} dr \;
  \phi_t(f - r; r) P(f - r, t) -
  \int^f_{f - 1} dr \;
  \phi_t(f; -r) P(f, t).
  \label{eq_master_eq_jump}
\end{equation}
Having showed what the integration limits \textit{should look like} we come back
to the issue mentioned before. Nowhere in the literature I have come across
there has been an explicit account for these limits. From Kimura to Rice to even
Lassig, everyone seems to overlook what these integrations should be. In an
attempt to be more explicit and thorough, while still readable and intuitive I
wanted to show the limits for the integral in the master equation. Nevertheless
these limits will have to be modify for our approximations as we will see in a
bit.

As mentioned before, dealing with \eref{eq_master_eq_jump} is extremely
complicated due to its integro-differential nature. To simplify things now we
will make use of one of the most handful techniques ever given to physics - the
Taylor expansion. This expansion will allow us to cast the nasty
integro-differential equation into a partial differential equation of infinite
order. This again is difficult to deal with, but we can always compromise some
precision for simplicity by truncating the expansion up to some arbitrary order
that is easier to handle. In particular is common practice to truncate
expansions up to second order, giving in this case one of Motoo Kimura's most
fundamental results, and the basis of all of diffusion theory in population
genetics. To perform the expansion of the master equation we have to assume the
following:
\begin{itemize}
  \item Changes on $f$ occur via very small jumps; i.e. the transition
  probability $\phi_t(f; r)$ is sharply peaked as a function of $r$, but at the
  same time varies slowly enough with $f$. This is a reasonable assumption in
  the limit of large population sizes since changes in the allele frequency will
  be slow for this regime.
  \item The probability of an allele frequency $P(f, t)$ also varies slowly with
  $f$. This smoothness requirement for the distribution is necessary since the
  expansion requires computing derivatives.
\end{itemize}
Both of these assumptions are necessary for the expansion to be valid at least
locally around the point we expand at.

\subsection{Issues with the integration limits}

As mentioned before, the integration limits on \eref{eq_master_eq_jump} must be
modified if we want to perform the Kramers-Moyal expansion. To show why this is
the case we will first attempt the expansion keeping the limits. Given the
assumptions we listed we can expand the first integrand in
\eref{eq_master_eq_jump} in powers of $f$ obtaining
\begin{equation}
  \begin{split}
    \ddt{P(f, t)} &= \int^f_{f - 1} dr \;
    \left[
    \phi_t(f; r) P(f, t) -
    {\partial \over \partial f} \left( \phi_t(f; r) P(f, t) \right) r \right. \\
    &+ \left.
    {1 \over 2} {\partial^2 \over \partial f^2}
    \left( \phi_t(f; r) P(f, t) \right) r^2 -
    {1 \over 3!} {\partial^3 \over \partial f^3}
    \left( \phi_t(f; r) P(f, t) \right) r^3 + \ldots
    \right] \\
    &-
    P(f, t) \int_{f - 1}^f dr \; \phi_t(f, -r).
  \end{split}
\end{equation}
If we now distribute the integral and write the general form of the expansion
we obtain
\begin{equation}
  \begin{split}
    \ddt{P(f, t)} &= P(f, t)\int^f_{f - 1} dr \; \phi_t(f; r) \\
    &+
    \sum_{k=1}^\infty {(-1)^k \over k!} {\partial^k \over \partial f^k}
    \left[
    P(f, t) \int_{f-1}^f dr \; r^k \phi_t(f; r)
    \right] \\
    &-
    P(f, t) \int_{f - 1}^f dr \; \phi_t(f, -r).
  \end{split}
\end{equation}
The problem with this expansion is that all integrals of the form
\begin{equation}
  \Phi(f, k) = \int_{f-1}^f dr \; r^k \phi_t(f; r),
  \text{ for } k \in \mathbb{N},
\end{equation}
have the wrong integration limits. Recall that we set these limits to cover the
entire domain of values that the jump size could take. Before the expansion
these limits allowed $\phi_t(f - r; r)$ to cover all jumps from any point
between zero and one towards $f$; after the expansion this is not true anymore.
For example if we were to substitute the lower integration limit into the
transition probability $\phi_t(f; r = f-1)$, the resulting final position $f + f
-1 = 2f -1$ would be outside of the range $[0, 1]$ for all $f < 1$. That is why
the Kramers-Moyal expansion is generally done for cases in which the boundaries
are irrelevant, for example Einstein's Brownian motion problem didn't have this
issue since the integration was taken from $-\infty$ to $\infty$.

So how do we overcome this issue? The way to overcome this difficulty is to
set the integration limits to some irrelevant boundary. But in order to do so
we need to define our transition probability $\phi_t(f; r)$ to be zero if the
jump falls outside of the range $[0, 1]$. In other words, we could define the
transition probability as a piecewise function
\begin{equation}
  \phi_t(f; r) =
  \begin{cases}
    \phi_t(f + r \mid f)& \text{for } (f + r), f \in [0, 1] \\
    0& \text{otherwise}
  \end{cases}.
\end{equation}
This definition allow us to send the integration limits of
\eref{eq_master_eq_jump} from $-\infty$ to $\infty$ to still cover all possible
jumps. With this definition we then have
\begin{equation}
  \begin{split}
    \ddt{P(f, t)} &= P(f, t)\int^\infty_{- \infty} dr \; \phi_t(f; r) \\
    &+
    \sum_{k=1}^\infty {(-1)^k \over k!} {\partial^k \over \partial f^k}
    \left[
    P(f, t) \int_{-\infty}^\infty dr \; r^k \phi_t(f; r)
    \right] \\
    &-
    P(f, t) \int_{-\infty}^\infty dr \; \phi_t(f, -r).
  \end{split}
\end{equation}
Notice that the first and third term on the right-hand side of the equation are
integrating the transition probability over all possible jumps. Because of the
change of integration limits we proposed, these two terms cancel each other and
we are left with
\begin{equation}
  \ddt{P(f, t)} = \sum_{k=1}^\infty {(-1)^k \over k!}
  {\partial^k \over \partial f^k}
  \left[
  P(f, t) \int_{-\infty}^\infty dr \; r^k \phi_t(f; r)
  \right].
  \label{eq_almost_km}
\end{equation}
Let us now define the moments of the jump size distribution $\phi_t(f; r)$ to be
\begin{equation}
  a^{(k)}(f, t) \equiv \int_{-\infty}^\infty dr \; r^k \phi_t(f; r),
  \label{eq_jump_mom}
\end{equation}
Then we can substitute this into \eref{eq_almost_km} to finally obtain the
Kramers-Moyal expansion of the master equation
\begin{equation}
  \ddt{P(f, t)} = \sum_{k=1}^\infty {(-1)^k \over k!}
  {\partial^k \over \partial f^k}
  \left[
  P(f, t) a^{(k)}(f, t)
  \right].
  \label{eq_km_expansion}
\end{equation}
Notice that on \eref{eq_transition_short_time} we used already the zero$\tth$
moment of the distribution; that's why the use of the notation $a^{(0)}(f, t)$.

So far we converted the complicated integro-differential form of the master
equation in \eref{eq_master_eq_jump} into an equally difficult partial
differential equation of infinite order. Therefore our ability to handle the
equation hasn't been simplified. In the next section we will deal with how to
actually improve our chances of making sense of this by truncating the expansion
up to second order.
