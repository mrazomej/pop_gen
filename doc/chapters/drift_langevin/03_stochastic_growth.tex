\subsubsection{Stochastic growth}

The example we just studied in the previous section hopefully gave us some
intuition for how to work with these Langevin equations. Now we are ready to
apply these stochastic equations to our evolutionary context. In particular our
objective is to model the stochasticity of cellular growth. At the cellular
scale that we have been thinking about for bacteria, what this means is that
not all bacteria divide synchronously at the exact same time. There is a
distribution of cell-cycle length that depends on things such as nutrient
availability and temperature. The result of this cell-to-cell variability in
cell-cycle length is that there is noise in the growth curves that must be
accounted for. But how large is this noise and what is its functional form? In
other words, what is the form of the $B(x, t)$ term in \eref{eq_langevin} when
thinking about how cells grow?

When we learn about bacteria growth we are usually told the fact that bacteria 
grow exponentially. This is a natural consequence of assuming that all bacteria
grow and divide at the same rate. So in the first division we pass from 1 to 2
bacteria, then we go from 2 to 4 and so on. This leads to an exponential growth
of the population as described by \eref{eq_expo_growth}.