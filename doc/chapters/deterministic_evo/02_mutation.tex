% !TEX root = ../../main.tex
\section{Mutation}

One of the ingredients for evolution to take place is the constant appearance
of genetic variability. After all, the raw material for evolution to act on is
the appearance of new mutations in the population. This is what we will model
now for our one-locus two-allele case. We will think of mutations between both
alleles $A$ and $a$ as a simple first-order chemical reaction of the form
\begin{equation}
  A \xrightleftharpoons[\mu_{a\rightarrow A}]
  {\,\mu_{A\rightarrow a}\,} a,
\end{equation}
where $\mu_{A\rightarrow a}$ is the mutation rate from $A$ to $a$ and 
$\mu_{a\rightarrow A}$ is the mutation rate from $a$ to $A$. A useful way to
interpret the mutation rate is as the probability of switching allele per unit
time. That means that on a small time window $\Dt$ a single organism has a
probability of switching from $A$ to $a$ of
\begin{equation}
  P(A \rightarrow a \mid \Dt) = \mu_{a\rightarrow A} \Dt.
\end{equation}
What this implies is that for this small time window every organism carrying
allele $A$ flips a coin with probability $\mu_{A\rightarrow a} \Dt$ of getting
heads. If the outcome of the coin is indeed heads, the allele is mutated,
otherwise it remains the same. So when writing a differential equation to model
this phenomena we must multiply the number of organisms by the mutation rate.
This results in a differential equation for the change in number of organisms
carrying $A$ of the form
\begin{equation}
  \dt{N_A} = 
  \underbrace{- \mu_{A\rightarrow a} N_A(t)}_{\text{loss } A \rightarrow a}
  \underbrace{- \mu_{a\rightarrow A} N_a(t)}_{\text{gain } a \rightarrow A},
\end{equation}
where, as indicated, we must account for number of organisms that ``exit'' the
$A$ allele state by mutating to $a$, and the number of organisms that ``enter''
the state by mutating from $a$ to $A$. We can write an equivalent equation for
$N_a$ where the signs would be simply flipped since every time an organism is
lost from allele $A$ it must be gained in allele $a$, and vice versa. This
implies that for this case where there is only mutation $N\tot$ does not change
over time since the number of organisms is assumed to be conserved.

Let us now write the dynamics that we actually care about, that is the allele
frequency $x(t)$. This takes the form
\begin{equation}
  \dt{x} = \dt{}\left( {N_A \over N\tot} \right) = {1 \over N\tot} \dt{N_A},
\end{equation}
where we took $N\tot$ out of the derivative since for this case we said it
doesn't change over time. Substituting the dynamics of $N_A$ and using the
definition of the allele frequency we obtain
\begin{equation}
  \dt{x} = - \mu_{A\rightarrow a} x + \mu_{a\rightarrow A} (1 - x),
\end{equation}
where we again suppressed the time dependence for notation simplicity. This
equation can also be solved analytically, but before getting to that solution
let's take a look at the steady state value. After all we know that these are
deterministic dynamics so the allele fraction will reach a unique point. The
steady state results in
\begin{equation}
  \mu_{A\rightarrow a} x = \mu_{a\rightarrow A} (1 - x).
\end{equation}
If we now solve for $x$ this results in
\begin{equation}
  x = {\mu_{a\rightarrow A} \over \mu_{a\rightarrow A} + \mu_{A\rightarrow a}}.
\end{equation}
So the equilibrium allele frequency for this case is given by the rate of how
often strains mutate from $a$ to $A$ divided by the sum of those rates. That
means that in the limit where it is much more likely to mutate from $a$ to $A$,
i.e. $\mu_{a\rightarrow A} \gg \mu_{A\rightarrow a}$ the allele frequency goes
to one, i.e. allele $A$ will go into fixation. The opposite would be true for a
much larger mutation rate from $A$ to $a$.

\subsection{Mutation-Selection Balance}

Now that we have modeled two of the three forces we will be using throughout
the notes let us try to put them together. Since both selection and mutation
are directional deterministic forces we can easily add them together for our
allele frequency dynamics and still obtain a deterministic answer