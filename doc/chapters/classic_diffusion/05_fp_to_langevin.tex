\subsection{From Langevin dynamics to Fokker-Planck equation}

Now that we built a description for the probability distribution over the
ensemble of trajectories that a Markovian stochastic process can follow we are
left with the task of how to define the Fokker-Plank equation coefficients from
the stochastic dynamics defined by the Langevin equation. Specifically in its
more general form we derived \eref{eq_fokker_planck} to depend on three things:
the probability distribution of allele frequencies $P(f, t)$, and the two jump
moments $a^{(1)}(f, t)$ and $a^{(2)}(f, t)$, where these moments are given by
\eref{eq_jump_mom}. A lot of the population genetics literature use a
non-mathematical argument to directly assign the values of these coefficients
given the either deterministic or stochastic nature of the terms. For example,
given the dynamics that we defined in \secref{seq_determ_mut} for the
deterministic form of mutation and selection (\eref{eq_sel_mut}) most texts
would directly assign these dynamics to be the first jump moment
$a^{(1)}(f,t)$. There is nothing intrinsically wrong with this approach, since
this is actually the correct answer, but it is not necessarily obvious that
this should be the case especially since \eref{eq_fokker_planck} requires us to
take a derivative with respect to $x$ of this coefficient multiplied by the
allele frequency distribution.

We will take a more formal approach and find how the terms in the Langevin
dynamics that we defined in \secref{sec_langevin_intro} connect to the
coefficients of the Fokker-Planck equation. To do so first we introduce a
quantity defined as
\begin{equation}
    \mathcal{A}^{(m)}(x; \tau, t) \equiv \int_{-\infty}^\infty dx\;
    (x - x')^m P(x, t + \tau \mid x', t), \; \text{for } m \geq 1.
    \label{eq_compute_jump_mom}
\end{equation}
This is nothing else than the average of $[x(t + \tau) - x(t)]^m$ with a sharp
initial condition $x(t) = x'$ for a Markov process that obeys the transition
probability distribution $P(x, t + \tau \mid x', t)$. Given that our Langevin
dynamics are Markov processes, the ensemble of trajectories must obey a
continuous master equation with such transition probability. Therefore
\eref{eq_compute_jump_mom} can be thought as the average displacement between
two time points $t$ and $t + \tau$ for our Langevin dynamics. In other words
$\mathcal{A}^{(m)}(x; \tau, t)$ can be expressed as
\begin{equation}
    \mathcal{A}^{(m)}(x; \tau, t) \equiv 
    \ee{\left[x(t + \tau) - x(t)\right]^m \mid x(t) = x'}.
    \label{eq_average_A}
\end{equation}
Just as we did to derive the master equation, for small $\tau$ we can rewrite
the conditional distribution $P(x, t + \tau \mid x', t)$ using
\eref{eq_transition_short_time}. Using this we can rewrite
\eref{eq_compute_jump_mom} as
\begin{equation}
    \mathcal{A}^{(m)}(x; \tau, t) = \int_{-\infty}^\infty dx\; (x - x')^m
    \underbrace{
    \left[ \delta(x - x') \left( 1 - a^{(0)}(x', t) \tau \right) +
    \phi_t(x \mid x')\tau \right]
    }_{P(x, t + \tau \mid x', t)}.
\end{equation}
Distributing the integrals results in
\begin{equation}
    \begin{aligned}
    \mathcal{A}^{(m)}(x; \tau, t) &= 
    \int_{-\infty}^\infty dx\; (x - x')^m \delta(x - x') \\
    &- \int_{-\infty}^\infty dx\; (x - x')^m \delta(x - x') a^{(0)}(x', t) 
    \tau \\ 
    &+ \int_{-\infty}^\infty dx\; (x - x')^m \phi_t(x \mid x') \tau.
    \end{aligned}
\end{equation}
For the first two terms we have that the $\delta$-function is one only when $x
= x'$ while the term $(x - x')^m$ is zero for this particular case. Therefore
both of these terms cancel and we are only left with
\begin{equation}
    \mathcal{A}^{(m)}(x; \tau, t) = \int_{-\infty}^\infty dx\; (x - x')^m 
    \phi_t(x \mid x') \tau.
\end{equation}
This is non other than the definition of the jump moments times $\tau$ given by
\eref{eq_jump_mom} since we defined $r \equiv x - x'$ and $\phi_t(x'; r) \equiv
\phi_t(x \mid x')$. Now we can see the connection; we started defining
$\mathcal{A}^{(m)}(x; \tau, t)$ to be the average displacement defined by our
Langevin dynamics over a small time interval $\tau$, and we found that this is
connected to the jump moments that define the coefficients of the Fokker-Planck
equation. So we have that
\begin{equation}
    \mathcal{A}^{(m)}(x; \tau, t) = \tau \cdot a^{(m)}(x, t).
\end{equation}
Given this result we can compute the jump moment by taking the derivative of
$\mathcal{A}^{(m)}$ with respect to $\tau$ and evaluating this derivative at
$\tau = 0$. This is
\begin{equation}
    a^{(m)}(x, t) = \left. {\partial \over \partial \tau} 
    \mathcal{A}^{(m)}(x; \tau, t) \right\vert_{\tau = 0}.
    \label{eq_A_tau_deriv}
\end{equation}
The reason we evaluate $\tau = 0$ is for convenience. In principle we could
evaluate the derivative for any value of $\tau$ since $a^{(m)}$ does not depend
on $\tau$, but zero is the one that simplifies things the most. Using the
definition of derivatives we then write \eref{eq_A_tau_deriv} as
\begin{equation}
    a^{(m)}(x, t) = \lim_{\Dt \rightarrow 0}
    \left.{\mathcal{A}^{(m)}(x, \tau + \Dt, t) - 
    \mathcal{A}^{(m)}(x, \tau, t) \over \Dt} \right\vert_{\tau = 0}.
\end{equation}
If we now substitute the equivalence of $\mathcal{A}^{(m)}$ with the average
displacement on our Langevin dynamics, i.e. \eref{eq_average_A} we obtain
\footnotesize
\begin{equation}
    a^{(m)}(x, t) = \lim_{\Dt \rightarrow 0} {1 \over \Dt}
    \left[
    \ee{\left[x(t + \tau + \Dt) - x(t)\right]^m \mid x(t) = x'} -
    \ee{\left[x(t + \tau) - x(t)\right]^m \mid x(t) = x'}
    \right]_{\tau = 0}.
    \label{eq_deriv_average}
\end{equation}
\normalsize
Now we use the condition $\tau = 0$ that we chose for simplification since the
second term in \eref{eq_deriv_average} cancels. We are then left with a simple
expression of the form
\begin{equation}
    a^{(m)}(x, t) = \lim_{\Dt \rightarrow 0}
    {\ee{\left[x(t + \Dt) - x(t)\right]^m \mid x(t) = x'} \over \Dt}.
\end{equation}
This result shows that given that our Langevin dynamics are a continuous Markov
process, we can compute the displacement dynamics for a small time interval,
elevate it to the $m\tth$ power, and then average it, to then compute the $m$
coefficient for the Fokker-Planck equation. Let's do this first for the first
coefficient $a^{(m)}(x, t)$.

\subsubsection{Directional coefficient of the Fokker-Planck equation}

For the first coefficient of the Fokker-Planck equation that defines the
directional terms of the dynamics we have
\begin{equation}
    a^{(1)}(x, t) = \lim_{\Dt \rightarrow 0}
    {\ee{x(t + \Dt) - x(t) \mid x(t) = x'} \over \Dt}.
\end{equation}
Recall that we defined the Langevin equations as a stochastic differential
equation given in \eref{eq_langevin}. Since $x(t)$ is the solution to this
differential equation we can use the fundamental theorem of calculus to rewrite
the average displacement as
\begin{equation}
    \ee{x(t + \Dt) - x(t) \mid x(t) = x'} =
    \ee{\left. \int_t^{t + \Dt} dt'\; {dx \over dt'} \right\vert x(t) = x'}.
\end{equation}
We then substitute \eref{eq_langevin} on the right-hand side to obtain
\begin{equation}
    \ee{x(t + \Dt) - x(t) \mid x(t) = x'} =
    \ee{\left.\int_t^{t + \Dt} dt'\; A(x, t') + B(x, t') \xi(t')\right\vert 
    x(t) = x'}.
\end{equation}
If we distribute the integral and use the linearity of expected values this
results in
\begin{equation}
    \ee{x(t + \Dt) - x(t) \mid x(t) = x'} =
    \int_t^{t + \Dt} dt\; A(x, t') + 
    \int_t^{t + \Dt} dt\; B(x, t') \ee{ \left.\xi(t') \right\vert x(t) = x'},
\end{equation}
where the only thing we needed to average over was the noise term $\xi(t)$. For
the Langevin equation we defined the average of this noise term to be zero
regardless of the initial position; that implies that the average displacement
is then simply given by
\begin{equation}
    \ee{x(t + \Dt) - x(t) \mid x(t) = x'} =
    \int_t^{t + \Dt} A(x, t').
\end{equation}
As a consequence the directional coefficient of the Langevin equation is then
of the form
\begin{equation}
    a^{(1)}(x, t) = \lim_{\Dt \rightarrow 0} {1 \over \Dt}
    \int_t^{t + \Dt} A(x, t').
\end{equation}
Since we are taking the limit when $\Dt \rightarrow 0$ we can approximate the
integral simply as
\begin{equation}
    a^{(1)}(x, t) \approx \lim_{\Dt \rightarrow 0}{1 \over \Dt} [A(x, t) \Dt] =
    A(x, t).
    \label{eq_approx_int_directional}
\end{equation}
We therefore mathematically justify the empirical argument commonly found in
the literature that the directional term associated with the Langevin equation
can be directly associated with the directional term of the Fokker-Planck
equation. In other words we showed that
\begin{equation}
    a^{(1)}(x, t) = A(x, t).
\end{equation}

\subsubsection{Diffusive coefficient of the Fokker-Planck equation}

To compute the second coefficient of the Fokker-Planck equation associated with
the diffusive random displacements we can follow the exact same recipe. First
we know that this coefficient is given by
\begin{equation}
    a^{(2)}(x, t) = \lim_{\Dt \rightarrow 0}
    {\ee{\left[x(t + \Dt) - x(t)\right]^2 \mid x(t) = x'} \over \Dt}.
\end{equation}
Again using the fundamental theorem of calculus gives us
\begin{equation}
    a^{(2)}(x, t) = \lim_{\Dt \rightarrow 0}
    {\ee{ \left. \left[ 
    \int_t^{t + \Dt} dt'\; {dx \over dt'}
    \right]^2 \right\vert x(t) = x'} \over \Dt}.
    \label{eq_diffusive_fp_lim}
\end{equation}
Substituting \eref{eq_langevin} on the expected value on right-hand side gives
\begin{equation}
    \ee{ \left. \left[ 
    \int_t^{t + \Dt} dt'\; {dx \over dt'}
    \right]^2 \right\vert x(t) = x'} =
    \ee{ \left. \left[ 
    \int_t^{t + \Dt} dt'\; A(x, t') + 
    \int_t^{t + \Dt} dt'\; B(x, t') \xi(t').
    \right]^2 \right\vert x(t) = x'},
\end{equation}
where we already distributed the integral. We now expand the binomial on the
right-hand side and use the linearity of expected values to obtain
\begin{equation}
    \begin{aligned}
    \ee{ \left. \left[ 
    \int_t^{t + \Dt} dt'\; A(x, t') + 
    \int_t^{t + \Dt} dt'\; B(x, t') \xi(t')
    \right]^2 \right\vert x(t) = x'} =
    \left(  
    \int_t^{t + \Dt} dt'\; A(x, t')
    \right)^2\\
    + 2 
    \int_t^{t + \Dt} dt'\; A(x, t')
    \int_t^{t + \Dt} dt'\; B(x, t') \ee{ \left.\xi(t')
    \right\vert x(t) = x'}\\
    \left(
    \int_t^{t + \Dt} dt'\; B(x, t') \ee{\left.\xi(t')
    \right\vert x(t) = x'}
    \right)^2,
    \end{aligned}
    \label{eq_int_binom}
\end{equation}
where again we used the fact that the only variable we can average over is the
white noise term $\xi(t)$. We can see that the second term on the right-hand
side cancels since the average of our stochastic variable is equal to zero. For
the two terms left we can rewrite the squared integrals as double integrals.
Let's take a look at each of these terms one at the time. First the integral 
involving the directional term takes the form
\begin{equation}
    \left(  
        \int_t^{t + \Dt} dt'\; A(x, t')
    \right)^2 =
    \int_t^{t + \Dt} dt'\; \int_t^{t + \Dt} dt''\;
    A(x, t') A(x, t'').
\end{equation}
Just as we did for \eref{eq_approx_int_directional} we can use the fact that we
are working at the limit where $\Dt \rightarrow 0$ to approximate the integral
as 
\begin{equation}
    \int_t^{t + \Dt} dt'\; \int_t^{t + \Dt} dt''\;
    A(x, t') A(x, t'') \approx \left[ A(x, t)  \Dt\right]^2.
    \label{eq_directional_int_approx}
\end{equation}
For the integral on \eref{eq_int_binom} involving the diffusive term we again
rewrite it as a double integral of the form
\begin{equation}
    \left(
    \int_t^{t + \Dt} dt'\; B(x, t') \ee{\left.\xi(t')
    \right\vert x(t) = x'}
    \right)^2 =
    \int_t^{t + \Dt} dt'\; \int_t^{t + \Dt} dt''\;
    B(x, t') B(x, t'') \ee{\left. \xi(t') \xi(t'') \right\vert x(t) = x}.
\end{equation}
In \secref{sec_langevin_intro} where we introduced the Langevin equation we
defined the second moment of our noise term as given in \eref{eq_noise_delta}.
Using this we evaluate one of the integrals first obtaining
\begin{equation}
    \int_t^{t + \Dt} dt'\; \int_t^{t + \Dt} dt''\;
    B(x, t') B(x, t'') \ee{\left. \xi(t') \xi(t'') \right\vert x(t) = x} =
    2D \int_t^{t + \Dt} dt'\; B(x, t')^2.
\end{equation}
Finally we can again approximate the integral as
\begin{equation}
    2D \int_t^{t + \Dt} dt'\; B(x, t')^2 \approx 2D B(x, t)^2 \Dt.
    \label{eq_diffusive_int_approx}
\end{equation}
Putting back \eref{eq_directional_int_approx} and
\eref{eq_diffusive_int_approx} into our computation of the second coefficient
of the Fokker-Planck equation in \eref{eq_diffusive_fp_lim} results in
\begin{equation}
    a^{(2)}(x, t) \approx \lim_{\Dt \rightarrow 0} {1 \over \Dt}
    \left[ A(x, t)^2\Dt^2 + 2D B(x, t)^2\Dt \right].
\end{equation}
Once we simplify the corresponding $\Dt$ terms and take the limit, the
directional term goes to zero and we are left with a Fokker-Planck diffusive
coefficient of the form
\begin{equation}
    a^{(2)}(x, t) = 2D B(x, t)^2.
\end{equation}