% !TEX root = ./main.tex
\section{Introduction}

\begin{enumerate}
    \item Evolution was first invented by physicists.
    \item What are the forces of evolution.
    \item What is population genetics.
    \item What is Kimura's diffusion theory.
    \item What is the benefit of having mathematical models of evolution.
\end{enumerate}

\subsection{Evolutionary dynamics}

\noindent
The concept of evolution is commonly associated with biology, but one may argue
that evolution on a broader sense has been a topic of physics for a much longer.
From Galileo's pioneering studies of the movement of objects on earth, to
Kepler's insight into planetary movements, physicists have been intrigue by the
time evolution of natural phenomena for centuries. But it wasn't until Newton's
Principia that calculus gave us a formal language to understand dynamical
processes. The mathematical theory for evolutionary dynamics is no exception.
Our objective will then be to write down differential equations for the time
trajectories of dynamical variables having to do with the genetic composition of
a population.

Since evolution is a population-level rather than a single-organism phenomena,
the relevant quantities we must keep track of in a mathematical theory of
evolution are relative abundances (frequencies) at this population level.
Specifically, a lot of the classic population genetics theory was casted in the
language of allele frequencies. An allele can be thought as a version of a gene.
In Mendel's original work he cross-pollinated different pea plants that
presented different physical characteristics (phenotypes) associated with
different alleles. If a pea plant carried a particular version of a gene, the
seeds presented a wrinkled surface. A different allele made seeds had a smooth
surface. Equivalently, there were different alleles for the other phenotypes
such as flower coloring or plant height.

The equivalent of the hydrogen atom in population genetics is the so-called
one-locus two-allele model in which a single gene (locus for its location on the
genome) presents only two possible versions (alleles). This simplified setup
will allow us to write down the dynamics of the relative frequency of both
alleles as a population evolves under different evolutionary forces. We then
imagine a population with $N$ organisms that carries two possible alleles, $A$
and $a$, for a particular gene of interest. The number of organisms carrying
allele $A$ or $a$ are given by $n$ and $(N - n)$, respectively, where $N$ is the
total population size. The population size can in principle take any value
between zero and infinity. But even if a population in which one of the two
alleles took over were to keep growing, we would not say that such a population
is evolving anymore. The population could still be growing, but nothing about
their characteristics (in this simplified model where we only track a single
gene) would change over time. That is why the quantity we care about is not the
absolute number of organisms carrying a specific allele, but their relative
frequency. We then define the allele frequency $x$ as
\begin{equation}
    x \equiv \frac{n}{N}.
\end{equation}
Our main objective will then be to write down dynamical equations for how this
allele frequency evolves over time. To write these dynamics we will need to work
in the regime where $N$ is very large. This will allow us to approximate both
$N$ and $x$ as continuous variables.

\subsection{Deterministic vs. stochastic evolutionary forces, and diffusion 
theory}

Throughout this work we will deal with three evolutionary forces: natural
selection, mutation, and genetic drift. These are force-fields in the very real
sense that there is a force associated with every point in our abstract space of
allele frequencies; just as there is a gravitational force associated with every
position nearby an object with mass. A schematic of this force field is depicted
in Firgure~\ref{fig01:diffusion_basics}(A). Following up with this gravity
analogy, two of the forces --selection and mutation-- are deterministic forces,
meaning that given certain initial conditions for the allele composition of a
population, the dynamics always reach the same end point if only these forces
are included. For example, if allele $A$ confers higher fitness, as long as $x(t
= 0) > 0$, allele $A$ will always take over the population if selection is the
only force present. This means we can write ordinary differential equations for
the allele frequency $\dot{x}$ that include mutation and selection.

Genetic drift, the process through which allele frequency changes due to random
sampling of alleles, is by definition a stochastic force. Analogous to the
stochastic forces that kick around a brownian particle undergoing a random walk,
the very nature of these random kicks can only be described probabilistically as
every displacement will be different. Mathematically, this implies that we can
either write down a stochastic differential equation, or, rather than writing
down an equation to describe how the allele frequency $x$ changes over time, we
can write down the dynamics for how the probability distribution of this allele
frequency $P(x, t)$ changes over time.

Diffusion theory, the formulation of evolutionary dynamics we focus on this
paper, brings together all forces under the same theoretical umbrella. By
describing the time evolution of the probability distribution, we can write down
a partial differential equation analogous to the diffusion-advection equation in
real space. As we will see later on, the major mathematical difference between
traditional diffusion and allele-frequency diffusion is the fact that the
stochastic forces depend on the allele frequency. Say this differently, the
diffusion coefficient in the abstract space of allele frequencies is not a
constant, but rather a function of the position in this space. Although
seemingly subtle, this complication makes the derivation and analysis of the
evolutionary dynamics significantly more difficult. To gain intuition into this
effect we will derive the partial differential equation (known as the
Fokker-Planck equation in the physics literature) from a discrete master
equation approach.

\begin{figure}[h!]
	\centering \includegraphics
  {../../fig/spread_the_butter/fig01_diffusion_basics.pdf}
  \caption{\textbf{Diffusion theory as a framework to put together all
  evolutionary forces}. (A) Schematic representation of the deterministic
  (mutation and selection) and stochastic (drift) force fields in the allele
  frequency space. The arrows show the direction of the expected change in
  allele frequency with the length proportional to the magnitude of the change.
  The curves show two different allele frequency probability distributions $P(x,
  t)$. (B) Discretization of the allele distribution into bins of width $\Delta
  x$. This discretization allows us to manipulate the continuous probability
  density function $P(x, t)$ as a discrete probability mass function $p(x, t)$.
  (C) Probability mass flow on a small time window $\Delta t$. Under the
  assumption that $\Delta t$ is sufficiently small we only consider the two
  contiguous bins to $x$, $x+\Delta x$ and $x-\Delta x$.}
  \label{fig01:diffusion_basics}
\end{figure}