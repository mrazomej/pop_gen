% !TEX root = ./main.tex
\section{General diffusion equation}

Our introduction to diffusion theory begins with the derivation of the partial
differential equation that defines the time evolution of the allele frequency
probability distribution $P(x, t)$. The approach we will consider here is a
discretization of the otherwise continuous probability density function.
Figure~\ref{fig01:diffusion_basics}(B) depicts the idea behind this
discretization where we divide the distributions into bins of some width $\Delta
x$. Since $P(x, t)$ is a \textit{probability density function}, i.e., a
probability function for continuous random variables, evaluating this density
for any particular value of the allele frequency returns zero. Only intervals of
the form $P(x - \delta x \leq x \leq x + \delta x)$ have meaningful
probabilities. This is because $P(x, t)$ has units of inverse $x$ (in our case,
inverse allele frequency), but probability has no dimensions. So by computing
the interval via an integration of the form
\begin{equation}
    P(x - \delta x \leq x \leq x + \delta x) =
    \int_{x - \delta x}^{x + \delta x} dx \; P(x, t),
\end{equation}
we multiply the density $P(x, t)$ (units of inverse allele frequency) with the
infinitesimal width $dx$ (units of allele frequency), recovering a dimensionless
quantity. Our discretization of this distribution then takes the form
\begin{equation}
    p(x, t) \equiv P(x - \frac{\Delta x}{2} \leq x \leq x + \frac{\Delta x}{2}) 
    \approx \overbrace{P(x, t)}^{\text{height}} 
    \overbrace{\Delta x}^{\text{width}},
\label{eq:mass_vs_density}
\end{equation}
where we define $p(x, t)$ to be a \textit{probability mass function}, i.e., a
probability function for discrete random variables. What we are doing is
approximating the integral as a rectangle of height $P(x, t)$ and a width $dx$.
The smaller $dx$ the more accurate our approximation gets.

Before jumping into the modeling of each of the individual evolutionary forces,
we will begin our analysis with an abstract transition probability that captures
the effect of all forces all together. This will allow us to gain intuition on
how to implement each of the particular forces in the following section. Let us
define a probability transition rate $\phi(x \mid x')$. The way to think about
this function is that, by taking into account all evolutionary forces involved,
$\phi(x \mid x')$ tells us what is the probability of transition from an allele
frequency $x'$ to a frequency $x$ per unit time. Given that $\phi(x \mid x')$ is
a rate (units of inverse time), we need to multiply is by a small window of time
$\Delta t$ in order to obtain a transition probability. For example, if we want
to compute what is the probability of transitioning from an allele frequency $x$
to an allele frequency $x + \Delta x$ in a small time window $\Delta t$, we must
compute
\begin{equation}
    P_{\text{transition}}(x \rightarrow x + \Delta x; \Delta t) = 
    \phi(x + \Delta x \mid x) \Delta t.
\end{equation}
With this in hand let's focus our attention to a particular frequency value $x$.
Figure~\ref{fig01:diffusion_basics}(C) shows the checks and balances we need to
consider for this allele frequency. What the schematic shows is that in a small
time window $\Delta t$ the ``amount of probability'' at $x$ can change due to
the outflow of probability mass from $x$ to the two contiguous bins $x + \Delta
x$ and $x - \Delta x$, and due to the inflow from those same two bins. We are
assuming that $\Delta t$ is sufficiently small such that we only need to
consider transitions from the nearest neighbors. Writing down this check and
balance equation we obtain
\begin{equation}
\begin{split}
    p(x, t + \Delta t) &= p(x, t)
    + \overbrace{p(x - \Delta x, t) \phi(x \mid x - \Delta x)\Delta t}
    ^{\text{inflow from the left}}
    + \overbrace{p(x + \Delta x, t) \phi(x \mid x + \Delta x)\Delta t}
    ^{\text{inflow from the right}}\\
    &- \overbrace{p(x, t) \phi(x - \Delta x \mid x) \Delta t}
    ^{\text{outflow to the left}}
    - \overbrace{p(x, t) \phi(x + \Delta x \mid x) \Delta t}
    ^{\text{outflow to the right}}.
\end{split}
\label{eq:inflow_outflow_diffusion}
\end{equation}
To make further progress, we need to rewrite the transition probability in terms
of the jump size $r \equiv x - x'$. In other words, instead of defining the
funciton with the initial and final position, we define the function with the
initial position and the size of the displacement. This is
\begin{equation}
    \phi(x \mid x') = \phi(x'; r).
\end{equation}
Using this allows us to rewrite \ref{eq:inflow_outflow_diffusion} as
\begin{equation}
\begin{split}
    p(x, t + \Delta t) = &p(x, t)
    + p(x - \Delta x, t) \phi(x - \Delta x; \Delta x)\Delta t
    + p(x + \Delta x, t) \phi(x + \Delta x; -\Delta x)\Delta t\\
    &- p(x, t) \phi(x; -\Delta x) \Delta t
    - p(x, t) \phi(x; \Delta x) \Delta t.
\end{split}
\label{eq:jump_diffusion}
\end{equation}
This equation needs a few rearrangements. First we send the first term on the
right hand side to the left, and divide both sides by $\Delta t$. This results
in
\begin{equation}
\begin{split}
    \frac{p(x, t + \Delta t) - p(x, t)}{\Delta t} &=
    p(x - \Delta x, t) \phi(x - \Delta x; -\Delta x)
    + p(x + \Delta x, t) \phi(x + \Delta x; \Delta x)\\
    &- p(x, t) \phi(x; -\Delta x) 
    - p(x, t) \phi(x; \Delta x).
\end{split}
\end{equation}
Taking the limit $\Delta t \rightarrow 0$ allows us to rewrite
\begin{equation}
\begin{split}
    \frac{d p(x, t)}{dt} &=
    p(x - \Delta x, t) \phi(x - \Delta x; -\Delta x)
    + p(x + \Delta x, t) \phi(x + \Delta x; \Delta x)\\
    &- p(x, t) \phi(x; -\Delta x) 
    - p(x, t) \phi(x; \Delta x).
\end{split}
\label{eq:master_eq_jump}
\end{equation}
This equation that describes the time evolution of the probability mass function
is usually called a master equation. As written right now,
Eq.~\ref{eq:master_eq_jump} depends on the unknown function $\phi(x; r)$ that
involves the influence of all evolutionary forces. The next manipulation of the
equation involves a Taylor expansion of the terms in Eq.~\ref{eq:master_eq_jump}
involving $x \pm \Delta x$ around $x$. As written right now we cannot Taylor
expand $p(x, t)$ since a probability mass function can only evaluated for the
\textit{discrete} values the random variable can take. It is at this point that
we use Eq.~\ref{eq:mass_vs_density} to convert back to the probability density
function. Substituting this in Eq.~\ref{eq:master_eq_jump} doesn't have any
effect given that the extra $\Delta x$ term that comes along with converting
from a mass to a density cancels from both sides, resulting in
\begin{equation}
\begin{split}
    \frac{d P(x, t)}{dt} &=
    P(x - \Delta x, t) \phi(x - \Delta x; -\Delta x)
    + P(x + \Delta x, t) \phi(x + \Delta x; \Delta x)\\
    &- P(x, t) \phi(x; -\Delta x) 
    - P(x, t) \phi(x; \Delta x).
\end{split}
\label{eq:master_eq_density}
\end{equation}
Now we can properly Taylor expand the terms involving $x \pm \Delta x$. This 
expansion takes the form
\begin{equation}
\begin{split}
    P(x \pm \Delta x, t) \phi(x \pm \Delta x; \mp \Delta x) &\approx
    P(x, t)\phi(x; \mp \Delta x)\\
    &+ \frac{d}{dx}
    \left[ P(x, t) \phi(x; \mp \Delta x)  \right] (\pm \Delta x)\\
    &+ \frac{1}{2} \frac{d^2}{d x^2}
    \left[ P(x, t) \phi(x; \mp \Delta x)  \right] (\pm \Delta x)^2.\\
\end{split}
\label{eq:taylor_expand_jump}
\end{equation}
Substituting Eq.~\ref{eq:taylor_expand_jump} into Eq.~\ref{eq:master_eq_density}
results in
\begin{equation}
\begin{split}
    \frac{\partial P(x, t)}{\partial t} &=
    P(x, t)\phi(x; - \Delta x)\\
    &+ \frac{\partial}{\partial x}
    \left[ P(x, t) \phi(x; - \Delta x)  \right] (\Delta x)\\
    &+ \frac{1}{2} \frac{\partial^2}{\partial x^2}
    \left[ P(x, t) \phi(x; - \Delta x) \right]   (\Delta x)^2\\
    &+ P(x, t) \phi(x; \Delta x)\\
    &+ \frac{\partial}{\partial x}
    \left[ P(x, t) \phi(x; \Delta x)  \right] (- \Delta x)\\
    &+ \frac{1}{2} \frac{\partial ^2}{\partial x^2}
    \left[ P(x, t) \phi(x; \Delta x)  \right] (- \Delta x)^2\\
    &- P(x, t) \phi(x; -\Delta x) 
    - P(x, t) \phi(x; \Delta x).
\end{split}
\end{equation}
This equation might look quite messy, but it can be enormously simplified by
cancelling terms and using the fact that derivatives are linear operators. Doing
so results in
\begin{equation}
    \frac{\partial P(x, t)}{\partial t} =
    \frac{\partial}{\partial x}
    \left[ P(x, t) 
    \left(\phi(x; - \Delta x) - \phi(x; \Delta x) \right) (\Delta x) 
    \right]
    + \frac{1}{2} \frac{\partial^2}{\partial x^2}
    \left[ P(x, t)
    \left(\phi(x; - \Delta x) - \phi(x; \Delta x)\right) (\Delta x)^2 \right].
\label{eq:PDE_diffusion}
\end{equation}

Let's look at the terms involving the transition probability. The first term on
the right hand side of the equation involves a term $\left(\phi(x; - \Delta x) -
\phi(x; \Delta x) \right) (\Delta x)$. This is nothing else than the mean
displacement. The way to see this is as follows: recall that in our discrete
approach for sufficiently small $\Delta t$ we only really have to consider the
three contiguous bins $x$ and $x \pm \Delta x$. To be consistent with the
notation in the population genetics literature, let us define $M(x)\Delta t$ as
the mean change in allele frequency during a small time window $\Delta t$. This
is analogous to the definition of $\phi(x \mid x')$ in which only the product of
$\phi(x \mid x')\Delta t$ satisfies the dimensionless nature of probabilities.
In other words, we can think of $M(x)$ as the mean rate with which the allele
frequency changes. The mean change in allele frequency is then computed as
\begin{equation}
M(x)\Delta t = \left\langle \Delta x \mid x \right\rangle_{\Delta x}\Delta t,
\end{equation}
where $\langle \cdot \rangle_{\Delta x}$ means to take the average over possible
jumps in allele frequency. What this equation says is that $M(x)\Delta t$
represents the expected shift in allele frequency on a small time window $\Delta
t$ given that the current allele frequency is $x$. Computing this average
displacement for our discrete case has only three options: a jump of size
$\Delta x$, a jump of size $- \Delta x$ and no jump. The average jump is then
given by.
\begin{equation}
    M(x)\Delta t = 
    \overbrace{(-\Delta x) \phi(x, - \Delta x)\Delta t}
    ^{\text{jump from $x$ to $x - \Delta x$}}
    + \overbrace{(\Delta x) \phi(x, + \Delta x)\Delta t}
    ^{\text{jump from $x$ to $x _ \Delta x$}}
    + \overbrace{(0) (1 - \phi(x, + \Delta x) - \phi(x, - \Delta x)\Delta t}
    ^{\text{no jump}}
\label{eq:Mx}
\end{equation}
From this we can easily see that
\begin{equation}
    M(x) = \left(\phi(x; + \Delta x) - \phi(x; - \Delta x) \right) (\Delta x).
\end{equation}
Notice this the change in the order in the terms compared with the first term in
Eq.~\ref{eq:PDE_diffusion}. This change of order will be carried as a minus sign
in our final result. In the same way let us define $V(x)\Delta t$ as the
variance in displacement during a small time window. This is written as
\begin{equation}
    V(x)\Delta t = \left[\langle (\Delta x)^2 \mid x \rangle_{\Delta x}
    - \left(\langle \Delta x \mid x \rangle_{\Delta x}\right)^2\right]\Delta t.
\end{equation}
As we showed in Eq.~\ref{eq:Mx}, the mean displacement scales with $\Delta x$.
This means that the second term on the right hand side of
Eq.~\ref{eq:variance_displacement} scales with $(\Delta x)^2$, which for very
small $\Delta x$ we will consider as negligible. With this simplification, the
variance in displacement during a small time window $\Delta t$ is then simply
given by
\begin{equation}
    V(x)\Delta t = \langle (\Delta x)^2 \mid x \rangle_{\Delta x} \Delta t.
\label{eq:variance_displacement}
\end{equation}
This can be computed again considering the three possible jumps as
\begin{equation}
    V(x)\Delta t = 
    \overbrace{(\Delta x)^2 \phi(x; \Delta x)\Delta t}
    ^{\text{jump from $x$ to $x + \Delta x$}}
    + \overbrace{(-\Delta x)^2 \phi(x; \Delta x)\Delta t}
    ^{\text{jump from $x$ to $x - \Delta x$}}
    + \overbrace{(0)^2 (1 - \phi(x, + \Delta x) - \phi(x, - \Delta x)\Delta t)
    \Delta t}
    ^{\text{no jump}}.
\end{equation}
From this we can see that 
\begin{equation}
    V(x) = \left(\phi(x; + \Delta x) - \phi(x; - \Delta x)\right) (\Delta x)^2.
\label{eq:Vx}
\end{equation}
Substituting Eqs.~\ref{eq:Mx}~and~\ref{eq:Vx} into Eq.~\ref{eq:PDE_diffusion} 
results in the diffusion-advection equation for allele frequency we were after
\begin{equation}
    \frac{\partial P(x, t)}{\partial t} = 
    - \frac{\partial}{\partial x}\left[ P(x, t) M(x) \right]
    + \frac{1}{2}\frac{\partial^2}{\partial x^2}\left[ P(x, t) V(x) \right].
\end{equation}
The reason for writing this general Fokker-Planck equation in terms of the mean
displacement rate $M(x)$ and the variance in the displacement rate $V(x)$ is
that the evolutionary are assigned to each of these functions. As we will see in
the next section, selection and mutation, being deterministic evolutionary
forces belong to the advection term, and therefore they define $M(x)$, while
genetic drift, a stochastic force, defines the term $V(x)$.