% !TEX root = ./main.tex
\section{Natural selection (survival of the fittest)}
\label{sec_selection}
1859 marked the year that Charles Darwin shocked the world with the incredible
insights contained in his book \textit{On the origin of Species}. The complete
title of the book \textit{On the Origin of Species by Means of Natural
Selection, or the Preservation of Favoured Races in the Struggle for Life}
captures perfectly Darwin's proposed mechanism for how evolution takes place.
Natural selection is intrinsically associated with the concept of fitness. The
phrase ``survival of the fittest'' first used by Darwin guided and still guides
the way that biologists think about the evolution of many organisms. But despite
the fact that fitness is part of the daily jargon of many biologists, it is a
subtle and highly debated concept. After all what defines the ability of an
organism to survive the challenges that surround them are completely context
dependent. Roughly speaking we can think of fitness as the ability of an
organism, or a population of organisms to survive and reproduce in the given
ecological niche they occupy. The term ecology has to be included because
fitness is a result of the interplay between organisms with their environment
including all biotic and abiotic interactions. 

It is common both in theory and in experiments to use the relative growth rates
of organisms, i.e. the speed at which they can reproduce and generate
offspring, as a proxy for fitness. This is a convenient approximation both
for experiments and for theory, but one should not lose track of the relevant
context dependence on the fitness. Just because redwoods have an average life
span of 500-700 years and a very low growth rate that doesn't mean they are not
fit. Having said that we will first begin with the simplest form of fitness,
i.e. frequency independent selection. The term frequency independence simply
refers to the assumption that the fitness of a particular allele does not
depend on the relative abundance of such allele. This assumption could break
down for cases such as some pathogenic bacteria that coordinate their attack
via cell-to-cell communication known as quorum sensing. 

\subsubsection{Frequency independent selection (haploids)}

Haploidy refers to the biological feature of having a single copy of a
chromosome. Given tthat many of the most interesting experiments in evolution
have been and are being done in haploid microorganisms, we will first focus our
efforts on these single-cell organisms. As bacteria divide into two cells they
segregate (split) one copy of the genome to each of the daughter cells
(extrachromosomal elements such as plasmids behave differently). This in
contrast with diploidy where organisms such as humans that receive one copy of
each chromosome form each parent. What haploidy means for our task of modeling
selection on a one-locus two-allele system is that we only need to focus on the
fitness value of each of the alleles without having to worry about combination
of alleles.

In order to write an analogous equation to \ref{eq:inflow_outflow_diffusion} we
need to define the transition probabilities to contiguous bins due to selection.
Let us define $f_A$ and $f_a$ to be the corresponding fitness of allele $A$ and
$a$, respectively. What these effective parameters represent is a net change in
the number of organisms with certain allele accounting for new organisms being
born and older organisms dying. This means that the net change in the number of
organisms carrying allele $A$ during a time window $\Delta t$ is given by
$\Delta N_A \approx f_A N_A \Delta t$. Likewise the net change of organisms
carrying allele $a$ is given by $\Delta N_a \approx f_a N_a \Delta t$. For a
large population and a sufficiently small time window we can assume that the
changes in population are negligible compared to the population size, i.e.,
$\Delta N_A, \Delta N_a \ll N_A + N_a$. This simplifying assumption allows us to
write the changes in the frequency of allele $A$ due to changes in the number of
organisms with allele $A$ as
\begin{equation}
    \Delta x(\Delta N_A) \approx \frac{\Delta N_A}{N_A + N_a} = f_A x \Delta t.
\end{equation}
Likewise, the change in $x$ due to changes in the number of organisms carrying
allele $a$ is given by
\begin{equation}
    \Delta x(\Delta N_a) \approx \frac{\Delta N_a}{N_A + N_a} = - f_a (1 - x) \Delta t,
\end{equation}
where the minus sign indicates that when $N_a$ increases while $N_A$ remains
fix, the change in the allele frequency of allele $A$ decreases. With this in
hand we can write the checks and balances equation for the probability mass
function $p(x, t)$ due to selection. For this we will assume that $f_A, f_a >
1$. This simply reflects the assumption that organisms with either allele are
viable, i.e., they can reproduce faster than they die. The resulting checks and
balances is of the form
\begin{equation}
\begin{split}
    p(x, t + \Delta t) &= p(x, t) \Delta t
    +\overbrace{[f_A (x - \Delta x) \Delta t] p(x - \Delta x, t)}
    ^{x - \Delta x \rightarrow x \text{ via change in } N_A}
    -\overbrace{[f_A x \Delta t] p(x, t)}
    ^{x \rightarrow x + \Delta x \text{ via change in } N_A}\\
    &+ \overbrace{[f_a (1 - (x + \Delta x)) \Delta t] p(x + \Delta x, t)}
    ^{x + \Delta x \rightarrow x \text{ via change in } N_a}
    -\overbrace{[f_a (1 - x) \Delta t] p(x, t)}
    ^{x \rightarrow x - \Delta x \text{ via change in } N_a}.
\end{split}
\end{equation}
As before, we send the first term on the right hand side to the left and divide
both sides by $\Delta t$. Upon taking the limit $\Delta t \rightarrow 0$ we
obtain the master equation for selection
\begin{equation}
    \frac{d p(x, t)}{d t} =[ f_A x+f_{a} (1 - x) ] p(x, t)
    +[f_A (x-\Delta x) ] p(x-\Delta x, t)
    +\left[f_{a}(1-(x+\Delta x))\right] p(x+\Delta x, t).
\end{equation}
The next step consists of performing the Taylor expansion of the terms involving
$x \pm \Delta x$. As in \ref{eq:master_eq_density} here is where we substitute
the probability mass function by the probability density function, which carries
a factor of $\Delta x$ with it. Since this factor is applied to both sides of
the equation, it ends up canceling out, resulting in
\begin{equation}
    \frac{d P(x, t)}{d t} =[ f_A x+f_{a} (1 - x) ] P(x, t)
    +[f_A (x-\Delta x) ] P(x-\Delta x, t)
    +\left[f_{a}(1-(x+\Delta x))\right] P(x+\Delta x, t).
\label{eq:master_fitness}
\end{equation}
The Taylor expansion of the second term on the right hand side of
Eq.~\ref{eq:master_fitness} is of the form
\begin{equation}
f_{A}(x-\Delta x) P(x-\Delta x, t) \approx f_{A} x P(x, t)
- f_A \frac{\partial}{\partial x} [x {P(x, t)]} \Delta x
+\frac{1}{2} f_{A} \frac{\partial^{2}}{\partial x^{2}}[x P(x, t)] \Delta x^{2}.
\end{equation}
Likewise, the Taylor expansion of the third on the right hand side of
Eq.~\ref{eq:master_fitness} is of the form
\begin{equation}
    f_{a} (1 - (x + \Delta x)) P(x + \Delta x, t) \approx 
    f_a(1 - x) P(x, t)
    + f_a \frac{\partial}{\partial x}[(1-x) P(x, t)] \Delta x
    + \frac{1}{2} f_a \frac{\partial^{2}}{\partial x^{2}}[(1-x) P(x, t)] 
    \Delta x^{2}.
\end{equation}
Substituting these Taylor expansions into Eq.~\ref{eq:master_fitness} results in
\begin{equation}
\begin{split}
\frac{\partial P(x, t)}{\partial t} = 
&- \left[f_{A} x + f_{a}(1-x)\right] P(x, t) \\
&+ f_{A} x P(x, t)-f_{A} \frac{\partial}{\partial x} \left[x P(x, t) \right] \Delta x \\
&+ f_{A} \frac{\partial^{2}}{\partial x^{2}} [x P(x, t)] \Delta x^{2} \\
&+ f_{a}(1-x) P(x, t)+f a \frac{\partial}{\partial x}[(1-x) P(x, t)] \Delta x \\
&+ f_{a} \frac{\partial^{2}}{\partial x^{2}}[(1-x) P(x, t)] \Delta x^{2}.
\end{split}
\end{equation}
Simplifying terms and using the linearity of derivatives allows us to write
\begin{equation}
\frac{\partial P(x, t)}{\partial t} =
- \frac{\partial}{\partial x}\left[\left(f_{A} x-f_{a}(1-x)\right)
P(x, t)\right] \Delta x 
+ \frac{1}{2} \frac{\partial^{2}}{\partial x^{2}}
\left[\left(f_{A} x + f_{a} (1-x) \right) P(x, t)\right] \Delta x^{2}
\label{eq:pde_fitness_2nd}
\end{equation}
Since the second term on the right hand side of Eq.~\ref{eq:pde_fitness_2nd} is
proportional to $\Delta x^2$, for very small $\Delta x$ we can consider this
term to be negligible. Note that for the first term on
Eq.~\ref{eq:pde_fitness_2nd} right hand side we can write
\begin{equation}
    \frac{\partial P(x, t)}{\partial t}=
    -\frac{\partial}{\partial x}
    \left[\left(f_A x \Delta x+f_{a}(1-x)(-\Delta x)\right) P(x, t)\right].
\end{equation}